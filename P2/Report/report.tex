\documentclass{report}

% Language setting
\usepackage[main=portuguese, english]{babel}
\usepackage{csquotes}

% Set page size and margins
\usepackage[a4paper,top=2cm,bottom=2cm,left=3cm,right=3cm,marginparwidth=1.5cm]{geometry}

% Useful packages
\usepackage{ulem}
\usepackage{parskip}
\usepackage{indentfirst}
\usepackage{setspace}
\usepackage{amsmath}
\usepackage{relsize}
\usepackage{array}

\usepackage{graphicx}
\usepackage{xcolor}
\usepackage{colortbl}
\usepackage{subfigure}
\usepackage{titlesec}
\usepackage[colorlinks=false, allbordercolors={0 0 0}, pdfborderstyle={/S/U/W 0.25}]{hyperref}
\usepackage[hypcap=true]{caption}
\usepackage{enumitem}
\usepackage{soul}

\usepackage[siunitx]{circuitikz}
\sisetup{output-decimal-marker={,}}

% Set section numbering from 1.1
\renewcommand{\thesection}{\arabic{section}.1}

\let\oldsection\section
\renewcommand\section{\clearpage\oldsection}

% Change section formatting
\titleformat{\section}
  {\fontsize{12}{15}\selectfont\bfseries}{\thesection}{1em}{}

% Configure indentations
\setlength{\parindent}{1.5cm}

\begin{document}

    \begin{titlepage}
        \centering
        
        \LARGE {Universidade Federal do Rio Grande do Sul \\ Escola de Engenharia}
    
        \begin{figure}[h!]
        \centering
        \subfigure
        {\includegraphics[width=0.35\linewidth]{images/logos/UFRGS.png}}
        \hspace{1cm}
        \subfigure
        {\includegraphics[width=0.3\linewidth]{images/logos/EE.png}}
        \end{figure}
    
        \LARGE {ENG10001 \\ Circuitos Elétricos I-C}
        
        \vfill
        {\noindent\hrulefill \\
        \bfseries \Huge{Trabalho Bônus 2} \\ \LARGE{Representação em Espaço de Estados} \\
        \noindent\hrulefill}
        
        \vfill
        {\LARGE Pedro Lubaszewski Lima (00341810) \\~\\ Turma A}
    
        \vfill
        {\LARGE 10 de janeiro de 2025}
        
    \end{titlepage}

        \renewcommand{\contentsname}{Sumário}
        \tableofcontents
        \clearpage
        \addtocontents{toc}{\protect\thispagestyle{empty}}

\section{Enunciado e Circuitos}

Este trabalho consiste em representar um dado circuito da \href{https://www.ece.ufrgs.br/~fetter/eng10001/listas/circuitos_2a_ordem.pdf}{lista principal de exercícios de circuitos de segunda ordem}
em espaços de estados. Esse circuito, bem como as saídas dele seriam sorteados de acordo com o número de matrícula. No entanto, para este trabalho, será feita a análise de todos os possíveis circuitos
sorteáveis, com todas as saídas sorteáveis. Todas as análises considerarão as condições iniciais em $ t_0 = 0^+\text{s} $ Abaixo são apresentados os circuitos com as suas respectivas saídas desejadas:

\begin{figure}[h!]
    \centering
    \begin{circuitikz}[scale=0.8]
        \draw (0,0) to[american current source, l=3<\ampere>] (0,3);
        \draw (0,3) -- (3,3)
              (0,0) -- (12,0);
        \draw (3,0) to[R, l=10<\ohm>, *-*] (3,3);
        \draw (3,3) to[ospst, l={$ t = 0\text{s} $}, *-*] (6,3);
        \draw (6,0) to[C, l=4<\farad>, *-*] (6,3);
        \draw (6.5,0.75) node[right]{$ - $}
              (6.5,1.5) node[right]{$ \text{V}_\text{C} $}
              (6.5,2.25) node[right]{$ + $};
        \draw (6,3) to[L, l=1<\henry>, *-*] (9,3);
        \draw [->, shorten >=1mm, shorten <=1mm] (8,2.6) -- (7,2.6) node[midway, below] {$ \text{I}_\text{L} $};
        \draw (9,0) to[R, l=5<\ohm>, *-*] (9,3);
        \draw (9.2,0.5) node[right]{$ - $}
              (9.2,1.5) node[right]{$ \text{V}_\text{R} $}
              (9.2,2.5) node[right]{$ + $};
        \draw (9,3) -- (12,3);
        \draw (12,0) to[american current source, l_=$ 4u(t)\text{A} $] (12,3);
    \end{circuitikz}
    \caption{\label{ckt:1} Circuito do Exercício 8.33}
\end{figure}

Nesse primeiro circuito, chamar-se-á de $ u = 4u(t)\text{A} $. A fonte mais à esquerda passa a não afetar o circuito em $ t \ge t_0 $. Além disso, $ x_1 = V_C $ e $ x_2 = I_L $.
Por fim, $ y_1 = V_C $, $ y_2 = I_L e $ $ y_3 = V_R $.

\begin{figure}[h!]
    \centering
    \begin{circuitikz}[scale=0.8]
        \draw (0,3) to[american voltage source, l_=80<\volt>] (0,0);
        \draw (0,3) to[R, l=10<\ohm>] (3,3);
        \draw (2.5,2.8) node[below]{$ - $}
              (1.5,2.8) node[below]{$ \text{V}_\text{R} $}
              (0.5,2.8) node[below]{$ + $};
        \draw (0,0) -- (15,0);
        \draw (3,0) to[R, l=10<\ohm>, *-*] (3,3);
        \draw (6,0) to[C, l=250<\micro\farad>, *-*] (6,3);
        \draw (6.5,0.75) node[right]{$ - $}
              (6.5,1.5) node[right]{$ \text{V}_\text{C} $}
              (6.5,2.25) node[right]{$ + $};
        \draw (3,3) -- (6,3);
        \draw (6,3) to[cspst, l={$ t = 0\text{s} $}, *-*] (9,3);
        \draw (9,0) to[L, l=16<\milli\henry>, *-*] (9,3);
        \draw [->, shorten >=1mm, shorten <=1mm] (9.4,2) -- (9.4,1) node[midway, right] {$ \text{I}_\text{L} $};
        \draw (9,3) to[ospst, l={$ t = 0\text{s} $}, *-*] (12,3);
        \draw (12,0) to[R, l=1<\kilo\ohm>, *-*] (12,3);
        \draw (12,3) -- (15,3);
        \draw (15,0) to[american current source, l_=10<\ampere>] (15,3);
    \end{circuitikz}
    \caption{\label{ckt:2} Circuito do Exercício 8.38.1}
\end{figure}

No segundo circuito, chamar-se-á de $ u = 80\text{V} $. A justificativa é igual àquela do exercício anterior, a fonte da direita não afeta mais o circuito em $ t \ge t_0 $, tempo de interesse. Além disso, $ x_1 = V_C $ e $ x_2 = I_L $.
Por fim, $ y_1 = V_C $, $ y_2 = I_L e $ $ y_3 = V_R $.

\begin{figure}[h!]
    \centering
    \begin{circuitikz}[scale=0.8]
        \draw (0,0) to[C, l=$ \frac{1}{3}\text{F} $] (0,6);
        \draw (0.5,2) node[right]{$ - $}
              (0.5,3) node[right]{$ \text{V}_\text{C} $}
              (0.5,4) node[right]{$ + $};
        \draw (0,0) -- (3,0)
              (0,6) -- (3,6);
        \draw (3,3) to[L, l=$ \frac{3}{4}\text{H} $, *-*] (3,6);
        \draw [->, shorten >=1mm, shorten <=1mm] (3.4,5) -- (3.4,4) node[midway, right] {$ \text{I}_\text{L} $};
        \draw (3,0) to[R, l=5<\ohm>, *-*] (3,3);
        \draw (3.2,0.5) node[right]{$ - $}
              (3.2,1.5) node[right]{$ \text{V}_\text{R} $}
              (3.2,2.5) node[right]{$ + $};
        \draw (6,6) to[american current source, l_=$ 2u(-t)\text{A} $] (3,6);
        \draw (6,6) -- (6,0);
        \draw (3,3) to[R, l=10<\ohm>, *-*] (6,3);
        \draw (3,0) to[R, l=10<\ohm>] (6,0);
    \end{circuitikz}
    \caption{\label{ckt:3} Circuito do Exercício 8.38.2}
\end{figure}

No terceiro, a partir de $ t = t_0 $, não há mais fontes de alimentação. Por conta disso, não há entradas para a representação
em espaço de estados. Além disso, $ x_1 = V_C $ e $ x_2 = I_L $. Por fim, $ y_1 = V_C $, $ y_2 = I_L e $ $ y_3 = V_R $.

\clearpage
\begin{figure}[h!]
    \centering
    \begin{circuitikz}[scale=0.8]
        \draw (0,3) to[american voltage source, l_=$ 60u(t)\text{V} $] (0,0);
        \draw (0,3) to[R, l=30<\ohm>] (3,3);
        \draw (2.5,2.8) node[below]{$ - $}
              (1.5,2.8) node[below]{$ \text{V}_\text{R} $}
              (0.5,2.8) node[below]{$ + $};
        \draw (0,0) -- (9,0);
        \draw (3,0) to[R, l=20<\ohm>, *-*] (3,3);
        \draw (3,3) to[C, l=$\num{0,5}\text{F}$] (6,3);
        \draw (5.2,2.5) node[below]{$ - $}
              (4.5,2.5) node[below]{$ \text{V}_\text{C} $}
              (3.8,2.5) node[below]{$ + $};
        \draw (6,3) to[L, l=$\num{0,25}\text{H}$] (9,3);
        \draw [->, shorten >=1mm, shorten <=1mm] (7,2.6) -- (8,2.6) node[midway, below] {$ \text{I}_\text{L} $};
        \draw (9,3) to[american voltage source, l=$ 30u(t)\text{V} $] (9,0);
    \end{circuitikz}
    \caption{\label{ckt:4} Circuito do Exercício 8.39}
\end{figure}

Para finalizar, no último circuito, $ u_1 = 60u(t)\text{V} $ e $ u_2 = 30u(t)\text{V} $. Além disso, $ x_1 = V_C $ e $ x_2 = I_L $.
Ademais, $ y_1 = V_C $, $ y_2 = I_L e $ $ y_3 = V_R $.

\section{Representação dos Circuitos em Espaços de Estados}
\subsection{Circuito do Exercício 8.33}
Para analisar qualquer circuito, é mais fácil transformar as suas entradas em componentes genéricos já definidos anteriormente, tratando
as chaves da forma que ficarão após $ t = t_0 $. Com isso:
\begin{figure}[h!]
    \centering
    \begin{circuitikz}[scale=0.8]
        \draw (0,0) -- (6,0);
        \draw (0,0) to[C, l=4<\farad>] (0,3);
        \draw (0.5,0.75) node[right]{$ - $}
              (0.5,1.5) node[right]{$ \text{V}_\text{C} $}
              (0.5,2.25) node[right]{$ + $};
        \draw (0,3) to[L, l=1<\henry>] (3,3);
        \draw [->, shorten >=1mm, shorten <=1mm] (2,2.6) -- (1,2.6) node[midway, below] {$ \text{I}_\text{L} $};
        \draw (3,0) to[R, l=5<\ohm>, *-*] (3,3);
        \draw (3.2,0.5) node[right]{$ - $}
              (3.2,1.5) node[right]{$ \text{V}_\text{R} $}
              (3.2,2.5) node[right]{$ + $};
        \draw (3,3) -- (6,3);
        \draw (6,0) to[american current source, l_=u] (6,3);
    \end{circuitikz}
    \caption{\label{ckt:1_generic} Circuito Genérico do Exercício 8.33}
\end{figure}

Com o circuito bem simplificado acima, pode-se aplicar nós para começar a sua análise:
$$ u = I_R + I_L $$
$$ \Rightarrow I_R = u - I_L $$
$$ \Rightarrow I_R = u - x_2 $$

Como $ V_R = 5\Omega \cdot I_R $,
$$ V_R = 5\Omega \cdot (u - x_2) $$
$$ \Rightarrow V_R = 5\Omega \cdot u - 5\Omega \cdot x_2 $$
\begin{equation}
    \label{eq:8.33_y3}
    \centering
    \Rightarrow y_3 = - 5x_2 + 5u
\end{equation}

Além disso, as outras saídas são as próprias variáveis de estado:
\vspace*{-1.5\baselineskip}
\begin{center}
    \begin{align}
        \label{eq:8.33_y1}
        \Rightarrow y_1 &= x_1 \\
        \label{eq:8.33_y2}
        \Rightarrow y_2 &= x_2
    \end{align}
\end{center}

Agora, para analisar as derivadas dos estados, vale lembrar que:
$$ V_L = L \cdot \dot{I_L} $$
$$ \Rightarrow \dot{I_L} = \frac{V_L}{L} $$
$$ \Rightarrow \dot{x_2} = V_L $$
$$ I_C = C \cdot \dot{V_C} $$
$$ \Rightarrow \dot{V_C} = \frac{I_C}{C} $$
$$ \Rightarrow \dot{x_1} = \frac{I_C}{4} $$

Porém, como o capacitor e o indutor estão em série, a corrente sobre ambos é igual. Ou seja,
$$ I_C = I_L $$
$$ \Rightarrow I_C = x_2 $$
\begin{equation}
    \label{eq:8.33_x1'}
    \centering
    \Rightarrow \dot{x_1} = \frac{x_2}{4}
\end{equation}

Para descobrir a última derivada faltante, basta construir uma malha do lado esquerdo do circuito:
$$ -V_C -V_L + V_R = 0 $$
$$ \Rightarrow V_L = V_R - V_C $$

Como $ V_R = y_3 $ e já foi calculado antes em \ref{eq:8.33_y3} e $ V_C = x_1 $,
$$ \Rightarrow V_L = -5x_2 + 5u - x_1 $$
$$ \Rightarrow V_L = -x_1 -5x_2 + 5u $$
\begin{equation}
    \label{eq:8.33_x2'}
    \centering
    \Rightarrow \dot{x_2} = -x_1 -5x_2 + 5u
\end{equation}

Agora, sabendo o comportamento do circuito a partir do tempo de interesse, calcular-se-á os valores iniciais
dos estados para utilizar mais tarde na ferramenta matemática. Para tanto, o circuito antes da chave se manifestar,
em $ t < t_0 $, se comportava da seguinte forma:
\begin{figure}[h!]
    \centering
    \begin{circuitikz}[scale=0.8]
        \draw (0,0) to[american current source, l=3<\ampere>] (0,3);
        \draw (0,3) -- (6,3)
              (0,0) -- (9,0);
        \draw (3,0) to[R, l=10<\ohm>, *-*] (3,3);
        \draw (6,0) to[C, l=4<\farad>, *-*] (6,3);
        \draw (6.5,0.75) node[right]{$ - $}
              (6.5,1.5) node[right]{$ \text{V}_\text{C}(0^-) $}
              (6.5,2.25) node[right]{$ + $};
        \draw (6,3) to[L, l=1<\henry>] (9,3);
        \draw [->, shorten >=1mm, shorten <=1mm] (8.5,2.6) -- (7.5,2.6) node[midway, below] {$ \text{I}_\text{L}(0^-) $};
        \draw (9,0) to[R, l_=5<\ohm>] (9,3);
    \end{circuitikz}
    \caption{\label{ckt:1_0-} Circuito do Exercício 8.33 antes de $ t_0 $}
\end{figure}

Nesse instante, o capacitor pode ser considerado como carregado e em circuito aberto e o indutor como carregado em forma de curto-circuito,
ambos sem ferirem nenhuma LKT, nem LKC. Portanto, a corrente que passa pelo resistor de $ 5\Omega $ é calculável pelo simples divisor de corrente:
$$ I_{R_{5\Omega}} = 3\text{A} \cdot \frac{10\Omega}{10\Omega + 5\Omega} $$
$$ \Rightarrow I_{R_{5\Omega}} = 3\text{A} \cdot \frac{2}{3} $$
$$ \Rightarrow I_{R_{5\Omega}} = 2\text{A} $$

Em decorrência disso, o resto da corrente passa pelo outro resistor:
$$ \Rightarrow I_{R_{10\Omega}} = 1\text{A} $$

Agora, como é observado na figura \ref{ckt:1_0-},
$$ I_L(0^-) = -I_{R_{5\Omega}} $$
$$ I_L(0^-) = -2\text{A} $$

Como o resistor de $ 10\Omega $ está em paralelo com o capacitor, sabe-se que as suas tensões são iguais. Desta forma,
$$ V_C(0^-) = V_{R_{10\Omega}} $$
$$ \Rightarrow V_C(0^-) = 1\text{A} \cdot 10\Omega $$
$$ \Rightarrow V_C(0^-) = 10\text{V} $$

Como, no instante em que a chave é aberta, $ t = 0\text{s} $, e logo após, em $ t = 0^+\text{s} $, não há nenhuma fonte impulsiva,
nem alguma alteração que torne incoerente as LKT e LKC com essas condições calculadas acima, conclui-se que:
$$ V_C(0^+) = V_C(0^-) $$
$$ I_L(0^+) = I_L(0^-) $$

\vspace*{-1.5\baselineskip}
\begin{center}
    \begin{align}
        \label{eq:8.33_x1_0}
        \Rightarrow x_1(t_0) &= 10 \\
        \label{eq:8.33_x2_0}
        \Rightarrow x_2(t_0) &= -2
    \end{align}
\end{center}

Agrupando as equações \ref{eq:8.33_y3} à \ref{eq:8.33_x2_0}, obtém-se:
\begin{equation}
    \label{eq:8.33_sol}
    \centering
    \begin{split}
        \begin{bmatrix} \dot{x_1} \\ \dot{x_2} \end{bmatrix} &= \begin{bmatrix} 0 & \frac{1}{4} \\ -1 & -5 \end{bmatrix}
        \begin{bmatrix} x_1 \\ x_2 \end{bmatrix} + \begin{bmatrix} 0 \\ 5 \end{bmatrix} \begin{bmatrix} u \end{bmatrix} \\
        \begin{bmatrix} y_1 \\ y_2 \\ y_3 \end{bmatrix} &= \begin{bmatrix} 1 & 0 \\ 0 & 1 \\ 0 & -5 \end{bmatrix}
        \begin{bmatrix} x_1 \\ x_2 \end{bmatrix} + \begin{bmatrix} 0 \\ 0 \\ 5 \end{bmatrix} \begin{bmatrix} u \end{bmatrix} \\
        \begin{bmatrix} x_1(t_0) \\ x_2(t_0) \end{bmatrix} &= \begin{bmatrix} 10 \\ -2 \end{bmatrix}
    \end{split}
\end{equation}

\begin{center}
    \[ \scalebox{3}{$ * $} \]
  \end{center}

\subsection{Circuito do Exercício 8.38.1}
Para começar, analisar-se-á o comportamento genérico do circuito em $ t \ge t_0 $:
\begin{figure}[h!]
    \centering
    \begin{circuitikz}[scale=0.8]
        \draw (0,3) to[american voltage source, l_=u] (0,0);
        \draw (0,3) to[R, l=10<\ohm>] (3,3);
        \draw (2.5,2.8) node[below]{$ - $}
              (1.5,2.8) node[below]{$ \text{V}_\text{R} $}
              (0.5,2.8) node[below]{$ + $};
        \draw (0,0) -- (9,0);
        \draw (3,0) to[R, l=10<\ohm>, *-*] (3,3);
        \draw (6,0) to[C, l=250<\micro\farad>, *-*] (6,3);
        \draw (6.5,0.75) node[right]{$ - $}
              (6.5,1.5) node[right]{$ \text{V}_\text{C} $}
              (6.5,2.25) node[right]{$ + $};
        \draw (3,3) -- (9,3);
        \draw (9,0) to[L, l=16<\milli\henry>] (9,3);
        \draw [->, shorten >=1mm, shorten <=1mm] (9.4,2) -- (9.4,1) node[midway, right] {$ \text{I}_\text{L} $};
    \end{circuitikz}
    \caption{\label{ckt:2_generic} Circuito Genérico do Exercício 8.38.1}
\end{figure}

Pelas definições iniciais de saídas, já concluí-se que:

\vspace*{-1.5\baselineskip}
\begin{center}
    \begin{align}
        \label{eq:8.38.1_y1}
        y_1 &= x_1 \\
        \label{eq:8.38.1_y2}
        y_2 &= x_2
    \end{align}
\end{center}

Para a terceira saída, pode-se analisar a primeira malha da esquerda:
$$ -u + V_R + V_C = 0 $$
$$ \Rightarrow V_R = u - V_C $$
\begin{equation}
      \label{eq:8.38.1_y3}
      \centering
      \Rightarrow y_3 = -x_1 + u
\end{equation}

Agora, de forma similar ao exercício anterior,
$$ V_L = L \cdot \dot{I_L} $$
$$ \Rightarrow \dot{I_L} = \frac{V_L}{L} $$
$$ \Rightarrow \dot{x_2} = 62,\!5V_L $$
$$ I_C = C \cdot \dot{V_C} $$
$$ \Rightarrow \dot{V_C} = \frac{I_C}{C} $$
$$ \Rightarrow \dot{x_1} = 4000I_C $$

Com isso em mente, é aparente que o capacitor e o indutor estão em paralelo. Portanto:
$$ V_L = V_C $$
$$ \Rightarrow V_L = x_1 $$
\begin{equation}
      \label{eq:8.38.1_x2'}
      \centering
      \Rightarrow \dot{x_2} = 62,\!5x_1
\end{equation}

Agora, para $ I_C $, analisar-se-á o nó superior do circuito:
$$ I_R = I_{R_{10\Omega}} + I_C + I_L $$
$$ \Rightarrow I_C = I_R - I_{R_{10\Omega}} - I_L $$
$$ \Rightarrow I_C = \frac{-V_C + u}{10\Omega} - \frac{V_C}{10\Omega} - I_L $$
$$ \Rightarrow I_C = -\frac{x_1}{5} - x_2 +\frac{u}{10} $$
$$ \Rightarrow \dot{x_1} = 4000\left(-\frac{x_1}{5} - x_2 +\frac{u}{10}\right) $$
\begin{equation}
      \label{eq:8.38.1_x1'}
      \centering
      \Rightarrow \dot{x_1} = -800x_1 - 4000x_2 + 400u
\end{equation}

Com essa análise, partir-se-á para o estudo das condições iniciais de $ x_1 $ e $ x_2 $. Em $ t = 0^- $:
\begin{figure}[h!]
      \centering
      \begin{circuitikz}[scale=0.8]
          \draw (0,3) to[american voltage source, l_=80<\volt>] (0,0);
          \draw (0,3) to[R, l=10<\ohm>] (3,3);
          \draw (0,0) -- (6,0);
          \draw (3,0) to[R, l=10<\ohm>, *-*] (3,3);
          \draw (6,0) to[C, l=250<\micro\farad>] (6,3);
          \draw (6.5,0.75) node[right]{$ - $}
                (6.5,1.5) node[right]{$ \text{V}_\text{C}(0^-) $}
                (6.5,2.25) node[right]{$ + $};
          \draw (3,3) -- (6,3);
          \draw (10,0) to[L, l=16<\milli\henry>] (10,3);
          \draw [->, shorten >=1mm, shorten <=1mm] (10.4,2) -- (10.4,1) node[midway, right] {$ \text{I}_\text{L}(0^-) $};
          \draw (13,0) to[R, l_=1<\kilo\ohm>, *-*] (13,3);
          \draw (10,3) -- (16,3)
                (10,0) -- (16,0);
          \draw (16,0) to[american current source, l_=10<\ampere>] (16,3);
      \end{circuitikz}
      \caption{\label{ckt:2_0-} Circuito do Exercício 8.38.1 antes de $ t_0 $}
\end{figure}

Considerando essa situação anterior às chaves, observa-se que, considerando o capacitor em aberto por estar carregado,
$$ V_C(0^-) = 80\text{V} \cdot \frac{10\Omega}{10\Omega + 10\Omega} $$
$$ \Rightarrow V_C(0^-) = 80\text{V} \cdot \frac{1}{2} $$
$$ \Rightarrow V_C(0^-) = 40\text{V} $$

Para o indutor, considerando-o carregado como um curto-circuito, toda a corrente da fonte de corrente fluirá por ele:
$$ I_L(0^-) = 10\text{A} $$

Como não fontes impulsivas e, ao unir esses circuitos no circuito genérico \ref{ckt:2_generic}, não há conflitos em nenhuma LKT,
nem LKC, conclui-se que:
$$ V_C(0^+) = V_C(0^-) $$
$$ I_L(0^+) = I_L(0^-) $$

\vspace*{-1.5\baselineskip}
\begin{center}
    \begin{align}
        \label{eq:8.38.1_x1_0}
        \Rightarrow x_1(t_0) &= 40 \\
        \label{eq:8.38.1_x2_0}
        \Rightarrow x_2(t_0) &= 10
    \end{align}
\end{center}

Unindo os resultados das equações \ref{eq:8.38.1_y1} à \ref{eq:8.38.1_x2_0}, obtem-se:
\begin{equation}
      \label{eq:8.38.1_sol}
      \centering
      \begin{split}
          \begin{bmatrix} \dot{x_1} \\ \dot{x_2} \end{bmatrix} &= \begin{bmatrix} -800 & -4000 \\ 62,\!5 & 0 \end{bmatrix}
          \begin{bmatrix} x_1 \\ x_2 \end{bmatrix} + \begin{bmatrix} 400 \\ 0 \end{bmatrix} \begin{bmatrix} u \end{bmatrix} \\
          \begin{bmatrix} y_1 \\ y_2 \\ y_3 \end{bmatrix} &= \begin{bmatrix} 1 & 0 \\ 0 & 1 \\ -1 & 0 \end{bmatrix}
          \begin{bmatrix} x_1 \\ x_2 \end{bmatrix} + \begin{bmatrix} 0 \\ 0 \\ 1 \end{bmatrix} \begin{bmatrix} u \end{bmatrix} \\
          \begin{bmatrix} x_1(t_0) \\ x_2(t_0) \end{bmatrix} &= \begin{bmatrix} 40 \\ 10 \end{bmatrix}
      \end{split}
\end{equation}

\begin{center}
      \[ \scalebox{3}{$ * $} \]
\end{center}

\subsection{Circuito do Exercício 8.38.2}
Novamente, começar-se-á pelo comportamento genérico em $ t \ge t_0 $. A partir desse ponto, já não há mais fontes de alimentação, portanto,
as matrizes de entrada são todas nulas e não serão explicitamente representadas aqui:
\begin{figure}[h!]
      \centering
      \begin{circuitikz}[scale=0.8]
          \draw (0,0) to[C, l=$ \frac{1}{3}\text{F} $] (0,6);
          \draw (0.5,2) node[right]{$ - $}
                (0.5,3) node[right]{$ \text{V}_\text{C} $}
                (0.5,4) node[right]{$ + $};
          \draw (0,0) -- (3,0)
                (0,6) -- (3,6);
          \draw (3,3) to[L, l=$ \frac{3}{4}\text{H} $] (3,6);
          \draw [->, shorten >=1mm, shorten <=1mm] (3.4,5) -- (3.4,4) node[midway, right] {$ \text{I}_\text{L} $};
          \draw (3,0) to[R, l=5<\ohm>, *-*] (3,3);
          \draw (3.2,0.5) node[right]{$ - $}
                (3.2,1.5) node[right]{$ \text{V}_\text{R} $}
                (3.2,2.5) node[right]{$ + $};
          \draw (6,3) -- (6,0);
          \draw (3,3) to[R, l=10<\ohm>] (6,3);
          \draw (3,0) to[R, l=10<\ohm>] (6,0);
      \end{circuitikz}
      \caption{\label{ckt:3_generic} Circuito Genérico do Exercício 8.38.2}
  \end{figure}

De praxe, pela definição das saídas,

\vspace*{-1.5\baselineskip}
\begin{center}
    \begin{align}
        \label{eq:8.38.2_y1}
        y_1 &= x_1 \\
        \label{eq:8.38.2_y2}
        y_2 &= x_2
    \end{align}
\end{center}

Para as outras grandezas, vale realizar uma breve simplificada no circuito unindo os dois resistores da direita:
\begin{figure}[h!]
      \centering
      \begin{circuitikz}[scale=0.8]
          \draw (0,0) to[C, l=$ \frac{1}{3}\text{F} $] (0,6);
          \draw (0.5,2) node[right]{$ - $}
                (0.5,3) node[right]{$ \text{V}_\text{C} $}
                (0.5,4) node[right]{$ + $};
          \draw (0,0) -- (3,0)
                (0,6) -- (3,6);
          \draw (3,3) to[L, l=$ \frac{3}{4}\text{H} $] (3,6);
          \draw [->, shorten >=1mm, shorten <=1mm] (3.4,5) -- (3.4,4) node[midway, right] {$ \text{I}_\text{L} $};
          \draw (3,0) to[R, l=5<\ohm>, *-*] (3,3);
          \draw (3.2,0.5) node[right]{$ - $}
                (3.2,1.5) node[right]{$ \text{V}_\text{R} $}
                (3.2,2.5) node[right]{$ + $};
          \draw (3,0) -- (6,0)
                (3,3) -- (6,3);
          \draw (6,0) to[R, l=20<\ohm>] (6,3);
      \end{circuitikz}
      \caption{\label{ckt:3_generic_simp} Circuito Genérico Simplificado do Exercício 8.38.2}
  \end{figure}

Pela figura acima, é fácil enxergar que há um divisor de corrente logo acima da grandeza de saída de interesse:
$$ I_R = \frac{20\Omega}{5\Omega + 20\Omega}\cdot I_L $$
$$ \Rightarrow I_R = \frac{4}{5}I_L $$
$$ \Rightarrow V_R = 5\Omega \cdot \frac{4}{5}I_L $$
$$ \Rightarrow V_R = 4I_L $$
\begin{equation}
      \label{eq:8.38.2_y3}
      \centering
      \Rightarrow y_3 = 4x_2
\end{equation}

Para a variação nos estados:
$$ V_L = L \cdot \dot{I_L} $$
$$ \Rightarrow \dot{I_L} = \frac{V_L}{L} $$
$$ \Rightarrow \dot{x_2} = \frac{4}{3}V_L $$
$$ I_C = C \cdot \dot{V_C} $$
$$ \Rightarrow \dot{V_C} = \frac{I_C}{C} $$
$$ \Rightarrow \dot{x_1} = 3I_C $$

Como o capacitor e o indutor estão em anti-série, sabe-se que:
$$ I_C = -I_L $$
$$ \Rightarrow I_C = -x_2 $$
\begin{equation}
    \label{eq:8.38.2_x1'}
    \centering
    \Rightarrow \dot{x_1} = -3x_2
\end{equation}

Para a tensão sobre o indutor, basta construir uma malha:
$$ -V_C + V_L + V_R = 0 $$
$$ \Rightarrow V_L = V_C - V_R $$
$$ \Rightarrow V_L = V_C - 4\Omega\cdot I_L $$
$$ \Rightarrow V_L = x_1 - 4x_2 $$
$$ \Rightarrow \dot{x_2} = \frac{4}{3} \cdot(x_1 - 4x_2) $$
\begin{equation}
      \label{eq:8.38.2_x2'}
      \centering
      \Rightarrow \dot{x_2} = \frac{4}{3}x_1 -\frac{16}{3}x_2
\end{equation}

Agora, para as condições iniciais, partindo de $ t = 0^- $:
\begin{figure}[h!]
      \centering
      \begin{circuitikz}[scale=0.8]
          \draw (0,0) to[C, l=$ \frac{1}{3}\text{F} $] (0,6);
          \draw (0.5,2) node[right]{$ - $}
                (0.5,3) node[right]{$ \text{V}_\text{C}(0^-) $}
                (0.5,4) node[right]{$ + $};
          \draw (0,0) -- (3,0)
                (0,6) -- (3,6);
          \draw (3,3) to[L, l=$ \frac{3}{4}\text{H} $, *-*] (3,6);
          \draw [->, shorten >=1mm, shorten <=1mm] (3.4,5) -- (3.4,4) node[midway, right] {$ \text{I}_\text{L}(0^-) $};
          \draw (3,0) to[R, l=5<\ohm>, *-*] (3,3);
          \draw (6,6) to[american current source, l_=$ 2\text{A} $] (3,6);
          \draw (6,6) -- (6,0);
          \draw (3,3) to[R, l=10<\ohm>, *-*] (6,3);
          \draw (3,0) to[R, l=10<\ohm>] (6,0);
      \end{circuitikz}
      \caption{\label{ckt:3_0-} Circuito do Exercício 8.38.2 antes de $ t_0 $}
\end{figure}

Tratando os componentes que armazenam energia como carregados, o capacitor é um circuito aberto e indutor se comporta como
um curto-circuito. Logo, como não há outro caminho para a corrente da fonte percorrer,
$$ I_L(0^-) = 2\text{A} $$

Para determinar a tensão no capacitor, pode-se percorrer a malha esquerda novamente:
$$ -V_C(0^-) + V_L(0^-) + V_R(0^-) = 0 $$
$$ \Rightarrow V_C(0^-) = V_R(0^-) $$

Para descobrir $ V_R(0^-) $, pode-se utilizar o divisor de corrente entre os resistores para saber a corrente sobre o componente e
aplicar a Lei de Ohm:
$$ I_R(0^-) = \frac{10\Omega}{10\Omega + 5\Omega + 10\Omega}\cdot 2\text{A} $$
$$ \Rightarrow I_R(0^-) = \frac{4}{5}\text{A} $$
$$ V_R(0^-) = 5\Omega \cdot \frac{4}{5}\text{A} $$
$$ \Rightarrow V_R(0^-) = 4\text{V} $$

Portanto,
$$ V_C(0^-) = 4\text{V} $$

Como não há fontes impulsivas, nem chaves, não há nada que fira a LKC ou a LKT. Então:
$$ V_C(0^+) = V_C(0^-) $$
$$ I_L(0^+) = I_L(0^-) $$

Com isso,
\vspace*{-1.5\baselineskip}
\begin{center}
    \begin{align}
        \label{eq:8.38.2_x1_0}
        \Rightarrow x_1(t_0) &= 4 \\
        \label{eq:8.38.2_x2_0}
        \Rightarrow x_2(t_0) &= 2
    \end{align}
\end{center}

Agrupando \ref{eq:8.38.2_y1} à \ref{eq:8.38.2_x2_0}, obtem-se:
\begin{equation}
      \label{eq:8.38.2_sol}
      \centering
      \begin{split}
          \begin{bmatrix} \dot{x_1} \\ \dot{x_2} \end{bmatrix} &= \begin{bmatrix} 0 & -3 \\ \frac{4}{3} & -\frac{16}{3} \end{bmatrix}
          \begin{bmatrix} x_1 \\ x_2 \end{bmatrix} \\
          \begin{bmatrix} y_1 \\ y_2 \\ y_3 \end{bmatrix} &= \begin{bmatrix} 1 & 0 \\ 0 & 1 \\ 0 & 4 \end{bmatrix}
          \begin{bmatrix} x_1 \\ x_2 \end{bmatrix} \\
          \begin{bmatrix} x_1(t_0) \\ x_2(t_0) \end{bmatrix} &= \begin{bmatrix} 4 \\ 2 \end{bmatrix}
      \end{split}
\end{equation}

\begin{center}
      \[ \scalebox{3}{$ * $} \]
\end{center}

\subsection{Circuito do Exercício 8.39}
Partindo para a análise genérica em $ t \ge t_0 $:
\begin{figure}[h!]
      \centering
      \begin{circuitikz}[scale=0.8]
          \draw (0,3) to[american voltage source, l_=$ u_1 $] (0,0);
          \draw (0,3) to[R, l=30<\ohm>] (3,3);
          \draw (2.5,2.8) node[below]{$ - $}
                (1.5,2.8) node[below]{$ \text{V}_\text{R} $}
                (0.5,2.8) node[below]{$ + $};
          \draw (0,0) -- (9,0);
          \draw (3,0) to[R, l=20<\ohm>, *-*] (3,3);
          \draw (3,3) to[C, l=$\num{0,5}\text{F}$] (6,3);
          \draw (5.2,2.5) node[below]{$ - $}
                (4.5,2.5) node[below]{$ \text{V}_\text{C} $}
                (3.8,2.5) node[below]{$ + $};
          \draw (6,3) to[L, l=$\num{0,25}\text{H}$] (9,3);
          \draw [->, shorten >=1mm, shorten <=1mm] (7,2.6) -- (8,2.6) node[midway, below] {$ \text{I}_\text{L} $};
          \draw (9,3) to[american voltage source, l=$ u_2 $] (9,0);
      \end{circuitikz}
      \caption{\label{ckt:4_generic} Circuito Genérico do Exercício 8.39}
  \end{figure}

Sabe-se que, de cara, duas das saídas são:

\vspace*{-1.5\baselineskip}
\begin{center}
    \begin{align}
        \label{eq:8.39_y1}
        y_1 &= x_1 \\
        \label{eq:8.39_y2}
        y_2 &= x_2
    \end{align}
\end{center}

No único nó superior do circuito, observa-se que:
$$ I_R = I_{R_{20\Omega}} + I_L $$
$$ \Rightarrow I_{R_{20\Omega}} = I_R - I_L $$

Agora, com esse conhecimento, aplicando malhas na esquerda:
$$ -u_1 + V_R + V_{R_{20\Omega}} = 0 $$
$$ \Rightarrow -u_1 + 50\Omega\cdot I_R -20\Omega\cdot I_L = 0 $$
$$ \Rightarrow 50\Omega\cdot I_R = u_1 + 20\Omega\cdot I_L $$
$$ \Rightarrow I_R = \frac{1}{50}u_1 + \frac{2}{5}I_L $$
$$ \Rightarrow I_R = \frac{2}{5}x_2 + \frac{1}{50}u_1 $$

Pela Lei de Ohm, tem-se:
$$ V_R = 30\Omega \cdot I_R $$
\begin{equation}
      \label{eq:8.39_y3}
      \centering
      \Rightarrow y_3 = 12x_2 + \frac{3}{5}u_1
\end{equation}

Agora, para as variações de estados:
$$ V_L = L \cdot \dot{I_L} $$
$$ \Rightarrow \dot{I_L} = \frac{V_L}{L} $$
$$ \Rightarrow \dot{x_2} = 4V_L $$
$$ I_C = C \cdot \dot{V_C} $$
$$ \Rightarrow \dot{V_C} = \frac{I_C}{C} $$
$$ \Rightarrow \dot{x_1} = 2I_C $$

De cara, pela construção do circuito, percebe-se que o capacitor e o indutor estão
em série, ou seja,
$$ I_C = I_L $$
$$ \Rightarrow I_C = x_2 $$
\begin{equation}
      \label{eq:8.39_x1'}
      \centering
      \Rightarrow \dot{x_1} = 2x_2
\end{equation}

Para a última variação de estado, realizar-se-á a malha superior inteira do circuito:
$$ -u_1 + V_R + V_C + V_L + u_2 = 0 $$
$$ \Rightarrow V_L = u_1 - u_2 - V_R - V_C $$
$$ \Rightarrow V_L = u_1 - u_2 - 12x_2 - \frac{3}{5}u_1 - x_1 $$
$$ \Rightarrow V_L = - x_1 - 12x_2 + \frac{2}{5}u_1 - u_2 $$
$$ \Rightarrow \dot{x_2} = 4\left(-x_1 - 12x_2 + \frac{2}{5}u_1 - u_2\right) $$
\begin{equation}
      \label{eq:8.39_x2'}
      \centering
      \Rightarrow \dot{x_2} = -4x_1 - 48x_2 + \frac{8}{5}u_1 - 4u_2
\end{equation}

Partindo para as condições iniciais do circuito, em $ t = 0^- $, percebe-se que nenhum componente estava carregado até
esse instante. Por conta disso, é possível afirmar que:
$$ V_C(0^-) = 0\text{V} $$
$$ I_L(0^-) = 0\text{A} $$

Devido a não haver fontes impulsivas, nem chaves e não haver violações das leis fundamentais do circuito, os componentes que
se opõe à variação de tensão (capacitor) e à variação de corrente (indutor) mantêm esses valores até o instante de interesse:
$$ V_C(0^+) = V_C(0^-) $$
$$ I_L(0^+) = I_L(0^-) $$

\vspace*{-1.5\baselineskip}
\begin{center}
    \begin{align}
        \label{eq:8.39_x1_0}
        \Rightarrow x_1(t_0) &= 0 \\
        \label{eq:8.39_x2_0}
        \Rightarrow x_2(t_0) &= 0
    \end{align}
\end{center}

Portanto, unindo as afirmações \ref{eq:8.39_y1} à \ref{eq:8.39_x2_0}, obtem-se:
\begin{equation}
      \label{eq:8.39_sol}
      \centering
      \begin{split}
            \begin{bmatrix} \dot{x_1} \\ \dot{x_2} \end{bmatrix} &= \begin{bmatrix} 0 & 2 \\ -4 & -48 \end{bmatrix}
            \begin{bmatrix} x_1 \\ x_2 \end{bmatrix} + \begin{bmatrix} 0 & 0 \\ \frac{8}{5} & -4 \end{bmatrix} \begin{bmatrix} u_1 \\ u_2 \end{bmatrix} \\
            \begin{bmatrix} y_1 \\ y_2 \\ y_3 \end{bmatrix} &= \begin{bmatrix} 1 & 0 \\ 0 & 1 \\ 0 & 12 \end{bmatrix}
            \begin{bmatrix} x_1 \\ x_2 \end{bmatrix} + \begin{bmatrix} 0 & 0 \\ 0 & 0 \\ \frac{3}{5} & 0 \end{bmatrix} \begin{bmatrix} u_1 \\ u_2 \end{bmatrix} \\
          \begin{bmatrix} x_1(t_0) \\ x_2(t_0) \end{bmatrix} &= \begin{bmatrix} 0 \\ 0 \end{bmatrix}
      \end{split}
\end{equation}

\begin{center}
      \[ \scalebox{3}{$ * $} \]
\end{center}

\section{Simulações dos Resultados}
\subsection{Simulação do Exercício 8.33}
\subsection{Simulação do Exercício 8.38.1}
\subsection{Simulação do Exercício 8.38.2}
\subsection{Simulação do Exercício 8.39}
\section{Conclusões Finais}

\end{document}