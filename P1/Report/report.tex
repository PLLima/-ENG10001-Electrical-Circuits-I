\documentclass{report}

% Language setting
\usepackage[main=portuguese, english]{babel}
\usepackage{csquotes}

% Set page size and margins
\usepackage[a4paper,top=2cm,bottom=2cm,left=3cm,right=3cm,marginparwidth=1.5cm]{geometry}

% Useful packages
\usepackage{ulem}
\usepackage{parskip}
\usepackage{indentfirst}
\usepackage{setspace}
\usepackage{amsmath}
\usepackage{relsize}
\usepackage{array}

\usepackage{graphicx}
\usepackage{xcolor}
\usepackage{colortbl}
\usepackage{subfigure}
\usepackage{titlesec}
\usepackage[colorlinks=false, allbordercolors={0 0 0}, pdfborderstyle={/S/U/W 0.25}]{hyperref}
\usepackage[hypcap=true]{caption}
\usepackage{enumitem}
\usepackage{soul}

\usepackage[siunitx]{circuitikz}
\sisetup{output-decimal-marker={,}}

% Set section numbering from 1.1
\renewcommand{\thesection}{\arabic{section}.1}

\let\oldsection\section
\renewcommand\section{\clearpage\oldsection}

% Change section formatting
\titleformat{\section}
  {\fontsize{12}{15}\selectfont\bfseries}{\thesection}{1em}{}

% Configure indentations
\setlength{\parindent}{1.5cm}

\begin{document}

    \begin{titlepage}
        \centering
        
        \LARGE {Universidade Federal do Rio Grande do Sul \\ Escola de Engenharia}
    
        \begin{figure}[h!]
        \centering
        \subfigure
        {\includegraphics[width=0.35\linewidth]{images/logos/UFRGS.png}}
        \hspace{1cm}
        \subfigure
        {\includegraphics[width=0.3\linewidth]{images/logos/EE.png}}
        \end{figure}
    
        \LARGE {ENG10001 \\ Circuitos Elétricos I-C}
        
        \vfill
        {\noindent\hrulefill \\
        \bfseries \Huge{Trabalho Bônus 1} \\ \LARGE{Associação de Quadripolos} \\
        \noindent\hrulefill}
        
        \vfill
        {\LARGE Pedro Lubaszewski Lima (00341810) \\~\\ Turma A}
    
        \vfill
        {\LARGE 8 de dezembro de 2024}
        
    \end{titlepage}

        \renewcommand{\contentsname}{Sumário}
        \tableofcontents
        \clearpage
        \addtocontents{toc}{\protect\thispagestyle{empty}}

\section{Circuitos Sorteados}

Primeiramente, com o meu número de matrícula \textbf{0 0 3 4 1 8 1 0}, observa-se os seguintes dígitos sorteadores:

\begin{itemize}
  \item $ N_1 = 3$;
  \item $ N_2 = 4$;
  \item $ N_3 = 1$;
  \item $ N_4 = 8$;
  \item $ N_5 = 1$;
  \item $ N_6 = 0$.
\end{itemize}

A partir deles, sabe-se que os circuito a serem analisados são os seguintes:

\begin{itemize}
  \item Circuito de Entrada:
\end{itemize}

\begin{figure}[h!]
    \centering
    \begin{circuitikz}[scale=0.8]
        \draw
        (3,3) to[R, l=4<\ohm>] (0,3)
        to[american voltage source, l=12<\volt>] (0,0)
        to[R, l_=16<\ohm>] (3,0)
        to[R, l_=10<\ohm>, *-*] (3,3)
        to[R, l_=20<\ohm>] (6,3)
        (3,0) to[R, l_=5<\ohm>] (6,0)
        (6,0) to[american current source, l_=3<\ampere>, *-*] (6,3)
        (6,3) -- (8,3) (6,0) -- (8,0)
        (8,3) to[R, l=5<\ohm>, *-*] (8,0)
        (8,3) -- (10,3) node[ocirc=](A){} node[right]{A}
        (8,0) -- (10,0) node[ocirc](B){} node[right]{B}

    ; \end{circuitikz}
    \caption{\label{ckt:input_1} Circuito de Entrada 2}
\end{figure}

\begin{itemize}
  \item Primeira Topologia de Quadripolo:
\end{itemize}

\begin{figure}[h!]
    \centering
    \begin{circuitikz}[scale=0.8]
        \draw (0,0) node[ocirc=]{} node[above]{$ - $}
              (0,2) node[]{$ \text{V}_1 $}
              (0,4) node[ocirc=]{} node[below]{$ + $};
        \draw (0.06,4) to[R, l_=20<\ohm>] (3,4)
              [->, shorten >=1mm, shorten <=1mm] (0,4.3) -- (1,4.3) node[midway, above] {$ \text{I}_1 $};
        \draw (3,4) to[american controlled voltage source, l=$ \num{1,2} \text{V}_2 $] (3,2);
        \draw (3,2) -- (9,2);
        \draw (7,4) to[american controlled current source, l=$ 10 \text{I}_1 $, -*] (7,2)
              (7,4) -- (11,4);
        \draw (9,4) to[R, l=100<\ohm>, *-] (9,2);
        \draw [->, shorten >=1mm, shorten <=1mm] (11,4.3) -- (10,4.3) node[midway, above] {$ \text{I}_2 $};
        \draw (5,2) to[R, l=20<\ohm>, *-*] (5,0)
              (0.06,0) -- (11,0);
        \draw (11,0) node[ocirc=]{} node[above]{$ - $}
              (11,2) node[]{$ \text{V}_2 $}
              (11,4) node[ocirc=]{} node[below]{$ + $};
    \end{circuitikz}
    \caption{\label{ckt:quad_1_1} Topologia de Quadripolo 2 (\texttt{Q1})}
\end{figure}

\begin{itemize}
  \item Segunda Topologia de Quadripolo:
\end{itemize}

\begin{figure}[h!]
    \centering
    \begin{circuitikz}[scale=0.8]
        \draw (0,0) node[ocirc=]{} node[above]{$ - $}
              (0,2) node[]{$ \text{V}_1 $}
              (0,4) node[ocirc=]{} node[below]{$ + $};
        \draw (0.06,4) to[R, l_=1<\kilo\ohm>] (3,4)
              [->, shorten >=1mm, shorten <=1mm] (0,4.3) -- (1,4.3) node[midway, above] {$ \text{I}_1 $};
        \draw (3,4) to[R, l=100<\kilo\ohm>, *-*] (3,0)
              (2.8,3) node[left]{$ - $}
              (2.8,2) node[left]{$ \text{V}_\text{x} $}
              (2.8,1) node[left]{$ + $};
        \draw (3,4) to[R, l_=2<\kilo\ohm>] (6,4)
              (6,4) -- (11,4);
        \draw (6,4) to[R, l=50<\ohm>, *-] (6,2)
              to[american controlled voltage source, l=$ 10^5 \text{V}_\text{x} $, -*] (6,0);
        \draw (9,4) to[R, l=5<\kilo\ohm>, *-*] (9,0);
        \draw [->, shorten >=1mm, shorten <=1mm] (11,4.3) -- (10,4.3) node[midway, above] {$ \text{I}_2 $};
        \draw (0.06,0) -- (11,0)
              (11,0) node[ocirc=]{} node[above]{$ - $}
              (11,2) node[]{$ \text{V}_2 $}
              (11,4) node[ocirc=]{} node[below]{$ + $};
    \end{circuitikz}
    \caption{\label{ckt:quad_2_1} Topologia de Quadripolo 3 (\texttt{Q2})}
\end{figure}

\clearpage
\begin{itemize}
  \item Associação dos Quadripolos:
\end{itemize}

\begin{figure}[h!]
    \centering
    \begin{circuitikz}[scale=0.8]
        \draw (2,1) node[fourport, label={[anchor=center]:Q2}](Q2){}
              (0.06,0.425) -- (Q2.port1)
              (0.06,1.575) -- (Q2.port4)
              (0,0.425) node[ocirc=]{} node[above]{$ - $}
              (0,1) node[left]{$ \text{V}_{1_\text{EQ}} $}
              (0,1.575) node[ocirc=]{} node[below]{$ + $}
              [->, shorten >=1mm, shorten <=1mm] (0,1.875) -- (0.8,1.875) node[midway, above] {$ \text{I}_{1_\text{EQ}} $};
        \draw (6,1) node[fourport, label={[anchor=center]:Q1}](Q1){}
        (8.15,0.425) -- (Q1.port2)
        (8.15,1.575) -- (Q1.port3)
        (8.15,0.425) node[ocirc=]{} node[above]{$ - $}
        (8.15,1) node[right]{$ \text{V}_{2_\text{EQ}} $}
        (8.15,1.575) node[ocirc=]{} node[below]{$ + $}
        [->, shorten >=1mm, shorten <=1mm] (8.15,1.875) -- (7.15,1.875) node[midway, above] {$ \text{I}_{2_\text{EQ}} $};
      \draw (Q2.port2) -- (Q1.port1)
            (Q2.port3) -- (Q1.port4);
    \end{circuitikz}
    \caption{\label{ckt:quad_assoc} Associação dos Quadripolos \texttt{Q1} e \texttt{Q2}}
\end{figure}

\begin{itemize}
  \item Circuito de Saída:
\end{itemize}

\begin{figure}[h!]
    \centering
    \begin{circuitikz}[scale=0.8]
        \draw (0,0) node[ocirc=]{} node[left]{C}
              (0,4) node[ocirc=]{} node[left]{D};
        \draw (0.06,4) to[R, l_=100<\ohm>] (3,4)
              [->, shorten >=1mm, shorten <=1mm] (0,4.3) -- (1,4.3) node[midway, above] {I};
        \draw (3,4) to[american controlled voltage source, l=$ \num{0,5}\text{V} $, -*] (3,0);
        \draw (6,0) to[american controlled current source, l_=$ \num{0,5}\text{I} $, *-] (6,4);
        \draw (8.5,4) to[R, l=500<\ohm>, *-*] (8.5,0);
        \draw (0.06,0) -- (12,0)
              (6,4) -- (12,4)
              to[american controlled current source, l=$ 10^{-3}\text{V} $] (12,0)
              (11.25,1) node[left]{$ - $}
              (11.25,2) node[left]{V}
              (11.25,3) node[left]{$ + $};
    \end{circuitikz}
    \caption{\label{ckt:output_1} Circuito de Saída 1}
\end{figure}

\section{Circuito Equivalente de Thevénin da Entrada}

Partindo do circuito de entrada sorteado, pode-se adotar a estratégia de transformação de fontes repetidas vezes até
chegar-se no circuito equivalente de Thevénin:

\begin{center}
  \begin{circuitikz}[scale=0.8]
    \draw
    (3,3) to[R, l=4<\ohm>] (0,3)
    to[american voltage source, l=12<\volt>] (0,0)
    to[R, l_=16<\ohm>] (3,0)
    to[R, l_=10<\ohm>, *-*] (3,3)
    to[R, l_=20<\ohm>] (6,3)
    (3,0) to[R, l_=5<\ohm>] (6,0)
    (6,0) to[american current source, l_=3<\ampere>, *-*] (6,3)
    (6,3) -- (8,3) (6,0) -- (8,0)
    (8,3) to[R, l=5<\ohm>, *-*] (8,0)
    (8,3) -- (10,3) node[ocirc=](A){} node[right]{A}
    (8,0) -- (10,0) node[ocirc](B){} node[right]{B}

  ; \end{circuitikz}

  \[ \scalebox{3}{$ \equiv $} \]

  \begin{circuitikz}[scale=0.8]
    \draw
    (3,3) to[R, l=20<\ohm>] (0,3)
    to[american voltage source, l=12<\volt>] (0,0)
    -- (3,0)
    to[R, l_=10<\ohm>, *-*] (3,3)
    to[R, l_=20<\ohm>] (6,3)
    (3,0) to[R, l_=5<\ohm>] (6,0)
    (6,0) to[american current source, l_=3<\ampere>, *-*] (6,3)
    (6,3) -- (8,3) (6,0) -- (8,0)
    (8,3) to[R, l=5<\ohm>, *-*] (8,0)
    (8,3) -- (10,3) node[ocirc=](A){} node[right]{A}
    (8,0) -- (10,0) node[ocirc](B){} node[right]{B}

  ; \end{circuitikz}

\[ \scalebox{3}{$ \equiv $} \]

\begin{circuitikz}[scale=0.8]
  \draw
  (0,3) -- (4,3)
  (0,0) to[american current source, l_=$ \num{0,6}\text{A} $] (0,3)
  (0,0) -- (4,0)
  (2,3) to[R, l=20<\ohm>, *-*] (2,0)
  (4,3) to[R, l=10<\ohm>, *-*] (4,0)
  (4,3) to[R, l_=20<\ohm>] (7,3)
  (4,0) to[R, l_=5<\ohm>] (7,0)
  (7,0) to[american current source, l_=3<\ampere>, *-*] (7,3)
  (7,3) -- (9,3) (7,0) -- (9,0)
  (9,3) to[R, l=5<\ohm>, *-*] (9,0)
  (9,3) -- (11,3) node[ocirc=](A){} node[right]{A}
  (9,0) -- (11,0) node[ocirc](B){} node[right]{B}

; \end{circuitikz}

\[ \scalebox{3}{$ \equiv $} \]

\begin{circuitikz}[scale=0.8]
  \draw
  (0,3) -- (3,3)
  (0,0) to[american current source, l_=$ \num{0,6}\text{A} $] (0,3)
  (0,0) -- (3,0)
  (2,3) to[R, l=$ \frac{20}{3} \Omega $, *-*] (2,0)
  (3,3) to[R, l_=20<\ohm>] (6,3)
  (3,0) to[R, l_=5<\ohm>] (6,0)
  (6,0) to[american current source, l_=3<\ampere>, *-*] (6,3)
  (6,3) -- (8,3) (6,0) -- (8,0)
  (8,3) to[R, l=5<\ohm>, *-*] (8,0)
  (8,3) -- (10,3) node[ocirc=](A){} node[right]{A}
  (8,0) -- (10,0) node[ocirc](B){} node[right]{B}

; \end{circuitikz}

\clearpage
\[ \scalebox{3}{$ \equiv $} \]

\begin{circuitikz}[scale=0.8]
  \draw
  (0,3) to[R, l_=$ \frac{20}{3} \Omega $] (3,3)
  (0,3) to[american voltage source, l=4<\volt>] (0,0)
  (0,0) -- (3,0)
  (3,3) to[R, l_=20<\ohm>] (6,3)
  (3,0) to[R, l_=5<\ohm>] (6,0)
  (6,0) to[american current source, l_=3<\ampere>, *-*] (6,3)
  (6,3) -- (8,3) (6,0) -- (8,0)
  (8,3) to[R, l=5<\ohm>, *-*] (8,0)
  (8,3) -- (10,3) node[ocirc=](A){} node[right]{A}
  (8,0) -- (10,0) node[ocirc](B){} node[right]{B}

; \end{circuitikz}

\[ \scalebox{3}{$ \equiv $} \]

\begin{circuitikz}[scale=0.8]
  \draw
  (0,3) to[R, l_=$ \frac{95}{3} \Omega $] (3,3)
  (0,3) to[american voltage source, l=4<\volt>] (0,0)
  (0,0) -- (3,0)
  (3,0) to[american current source, l_=3<\ampere>, *-*] (3,3)
  (3,3) -- (5,3) (3,0) -- (5,0)
  (5,3) to[R, l=5<\ohm>, *-*] (5,0)
  (5,3) -- (7,3) node[ocirc=](A){} node[right]{A}
  (5,0) -- (7,0) node[ocirc](B){} node[right]{B}

; \end{circuitikz}

\[ \scalebox{3}{$ \equiv $} \]

\begin{circuitikz}[scale=0.8]
  \draw
  (0,3) -- (4,3)
  (0,0) to[american current source, l_=$ \frac{12}{95} \text{A} $] (0,3)
  (2,3) to[R, l=$ \frac{95}{3} \Omega $, *-*] (2,0)
  (0,0) -- (4,0)
  (4,0) to[american current source, l_=3<\ampere>, *-*] (4,3)
  (4,3) -- (6,3) (4,0) -- (6,0)
  (6,3) to[R, l=5<\ohm>, *-*] (6,0)
  (6,3) -- (8,3) node[ocirc=](A){} node[right]{A}
  (6,0) -- (8,0) node[ocirc](B){} node[right]{B}

; \end{circuitikz}

\[ \scalebox{3}{$ \equiv $} \]

\begin{circuitikz}[scale=0.8]
  \draw
  (0,0) to[american current source, l_=$ \frac{297}{95} \text{A} $] (0,3)
  (2,3) to[R, l=$ \frac{95}{22} \Omega $, *-*] (2,0)
  (0,3) -- (4,3) node[ocirc=](A){} node[right]{A}
  (0,0) -- (4,0) node[ocirc](B){} node[right]{B}

; \end{circuitikz}

\[ \scalebox{3}{$ \equiv $} \]

\begin{circuitikz}[scale=0.8]
  \draw
  (0,3) to[american voltage source, l=$ \frac{27}{2} \text{V} $] (0,0)
  (0,3) to[R, l_=$ \frac{95}{22} \Omega $] (3,3)
  (3,3) -- (4,3) node[ocirc=](A){} node[right]{A}
  (0,0) -- (4,0) node[ocirc](B){} node[right]{B}

; \end{circuitikz}

\[ \scalebox{3}{$ * $} \]

\end{center}

Assim, com a sequência ilustrada acima, chegou-se ao circuito equivalente de Thevénin da entrada
com $ V_{TH} = \frac{27}{2} \text{V} = 13,\!5 \text{V} $ e $ R_{TH} = \frac{95}{22} \Omega = 4,\!3\overline{18} \Omega $.

\section{Análise da Associação de Quadripolos}

\subsection{Representação dos Circuitos}

\subsection{Parâmetros do Quadripolo \texttt{Q1}}

\subsection{Parâmetros do Quadripolo \texttt{Q2}}

\subsection{União dos Quadripolos}

\section{Circuito Equivalente de Norton da Saída}

\section{Ganho de Tensão da Saída \texorpdfstring{\texttt{V\textsubscript{2}/V\textsubscript{1}}}{V2/V1}}

\section{Exemplos}

\begin{figure}[h!]
  \centering
  \includegraphics[width=.85\linewidth]{images/logos/UFRGS.png}
  \caption{\label{img:combination_dist} \textit{Violin Plot} de Consumo Médio}
\end{figure}

\begin{table}[h!]
  \centering
  \begin{tabular}{| c | c | c |}
      \hline
      \rowcolor{lightgray}
      \textbf{Modelo} & \textbf{Média dos MSE} & \textbf{Desvio Padrão dos MSE} \\
      \hline
      kNN & 5,4293 & 2,3616 \\
      \hline
      \textit{Random Forest} & 1,9517 & 1,1847 \\
      \hline
      Regressão Linear & 1,6631 & 0,9758 \\
      \hline
      Redes Neurais & 1,8377 & 1,0418 \\
      \hline
      SVM & 3,3739 & 1,3368 \\
      \hline
  \end{tabular}
  \caption{\label{table:model_summary} Médias e Desvios Padrões dos MSE}
\end{table}

\end{document}